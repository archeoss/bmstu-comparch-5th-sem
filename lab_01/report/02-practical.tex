\chapter{Практический раздел}

\section*{Создание нового модуля системы на кристале QSYS}

\begin{enumerate}
    \item Был создан новый модуль Qsys.
    \item Установлена частота внешнего сигнала синхронизации 50 000 000 Гц.
    \item Добавлен в проект модуль синхронизируемого микропроцессорного ядра Nios2.
    \item Добавлен в проект модуль ОЗУ программ и данных.
    \item Добавлены компоненты Avalon System ID, Avalon UART.
    \item Создана сеть синхронизации и сбоса системы.
    \item Сигналы TX и RX экспортированы во внешние порты.

    \item Назначены базовые адреса устройств.
\end{enumerate}

Итог выполненных действий показан на рисунке \ref{img:nios}.

\includeimage{nios}
{f} %
{h} %
{0.5\textwidth}
{Настройка модуля Nios2}


\section*{Назначение портами проекта контакты микросхемы}

Назначены котакты в соответствии с таблицей из методических указаний.

Таблица представлена на рисунке \ref{img:table}

\includeimage{table}
{f} %
{h} %
{0.75\textwidth}
{Таблица из методических указаний}
\clearpage

Был выполнен синтез проекта.

Результат представлен на рисунке \ref{img:pin_planer}.


\includeimage{pin_planer}
{f} %
{h} %
{0.75\textwidth}
{Модуль Pin Planner}
\clearpage

\section*{Сoздание проекта Nios2}

В файле  $hello\_world.c$ добавлен код эхо-программы приема-передачи по интерфейсу RS232.

Также был создан образ ОС HAL с драйверами устройств, используемых в аппаратном проекте.

Результат представлен на рисунке \ref{img:eclipse}.

\includeimage{eclipse}
{f} %
{h} %
{0.75\textwidth}
{Проект Nios2}
\clearpage

\section*{Подключение к ПК отладочной платы с ПЛИС EPC2C20 и вывод необходимого сообщения на экран}
К ПК была подключена отладочная плата с ПЛИС EPC2C20.

Была выполнена верификация проекта с использованием программы терминала. Доработан код проекта с использованием необходимых библиотек (представлены ниже).

$\#include "system.h"$

$\#include "altera\_avalon\_sysid\_qsys.h"$

$\#include "altera\_avalon\_sysid\_qsys\_regs.h"$

Для изменения SYSTEM\_ID были внесены измения в файл $system.h$.

Доработанный код проекта, а также вывод сообщения с номером группы (55) представлены на рисунке \ref{img:mod}.

\includeimage{mod}
{f} %
{h} %
{0.75\textwidth}
{Код программной части проекта}
\clearpage

\chapter{Вывод}

В ходе выполнения лабораторной работы были изучены основные понятия и принципы работы среды разработки Quartus II, а также основные понятия и принципы работы среды разработки Eclipse.

Был выполнен синтез проекта, создан проект Nios2, а также была выполнена верификация проекта с использованием программы терминала.

Также в ходе данной лабораторной работы были изучены основы построения микропроцессорных систем на ПЛИС, были выполнены проектирование и верификация системы с использованием отладочного комплекта  Altera DE1Board.

Поставленные цели лабораторной работы были достигнуты.
