\chapter{Теоретический раздел}

Функциональная схема разрабатываемой системы на кристалле представлена на рисунке \ref{img:scheme}.

\includeimage
{scheme} % Имя файла без расширения (файл должен быть расположен в директории inc/img/)
{f} %
{h} %
{0.75\textwidth} % Ширина рисунка
{Функциональная схема разрабатываемой системы на кристалле} % Подпись рисунка

Система на кристалле состоит из следующих блоков.

\begin{enumerate}
    \item Микропроцессорное ядро Nios II/e выполняет функции управления системой.
    \item Внутренняя оперативная память СНК, используемая для хранения программы
    управления и данных.
    \item Системная шина Avalon обеспечивает связность всех компонентов системы.
    \item Блок синхронизации и сброса обеспечивает обработку входных сигналов сброса и
    синхронизации и распределение их в системе. Внутренний сигнал сброса
    синхронизирован и имеет необходимую для системы длительность.
    \item Блок идентификации версии проекта обеспечивает хранение и выдачу уникального
    идентификатора версии, который используется программой управления при
    инициализации системы.
    \item Контроллер UART обеспечивает прием и передачу информации по интерфейсу RS232.
\end{enumerate}

\clearpage
