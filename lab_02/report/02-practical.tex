\chapter{Практический раздел}

\section{Задание 0}

Дизассемблированный код программы (\ref{lst:task0}):

\begin{lstlisting}[style={asm}, caption=Программа для задания 0, label={lst:task0}]
SYMBOL TABLE:
80000000 l    d  .text  00000000 .text
80000040 l    d  .data  00000000 .data
00000000 l    df *ABS*  00000000 test.o
00000008 l       *ABS*  00000000 len
00000004 l       *ABS*  00000000 enroll
00000004 l       *ABS*  00000000 elem_sz
80000040 l       .data  00000000 _x
8000000c l       .text  00000000 loop
8000003c l       .text  00000000 forever
80000000 g       .text  00000000 _start
80000060 g       .data  00000000 _end



Disassembly of section .text:

80000000 <_start>:
80000000:       00200a13                addi    x20,x0,2
80000004:       00000097                auipc   x1,0x0
80000008:       03c08093                addi    x1,x1,60 # 80000040 <_x>

8000000c <loop>:
8000000c:       0000a103                lw      x2,0(x1)
80000010:       002f8fb3                add     x31,x31,x2
80000014:       0040a103                lw      x2,4(x1)
80000018:       002f8fb3                add     x31,x31,x2
8000001c:       0080a103                lw      x2,8(x1)
80000020:       002f8fb3                add     x31,x31,x2
80000024:       00c0a103                lw      x2,12(x1)
80000028:       002f8fb3                add     x31,x31,x2
8000002c:       01008093                addi    x1,x1,16
80000030:       fffa0a13                addi    x20,x20,-1
80000034:       fc0a1ce3                bne     x20,x0,8000000c <loop>
80000038:       001f8f93                addi    x31,x31,1

8000003c <forever>:
8000003c:       0000006f                jal     x0,8000003c <forever>

Disassembly of section .data:
\end{lstlisting}
\clearpage

\section{Задание 1}

Листинг \ref{lst:task1} показывает код программы согласно варианту задания. (№18)
\lstinputlisting[style=asm, caption=Программа для задания 1 (вар.18), label=lst:task1]{../riscv-lab/src/task.s}
\clearpage

Псевдокод \ref{lst:task1_pseudocode} программы выше (\ref{lst:task1}):
\begin{algorithmic}\label{lst:task1_pseudocode}
\State $len \gets 9$ \Comment{Размер массива}
\State $enroll \gets 2$ \Comment{Количество обрабатываемых элементов за одну итерацию}
\State $elem\_sz \gets 4$ \Comment{Размер одного элемента массива}
\State $\_x \gets \texttt{0x1, 0x2, 0x3, 0x4, 0x8, 0x6, 0x7, 0x5, 0x4}$
\State $x1 \gets _x$
\State $x_{end} \gets x1 + len$
\State $x31 \gets x1[0]$
\State $x1 \gets x1 + 1$
\While {$x1 < x_{end}$}
    \State $x2 \gets x[0]$
    \State $x3 \gets x[1]$
    \If {$x2 > x31$}
    \State $x31 \gets x2$
    \EndIf
    \If {$x3 > x31$}
    \State $x31 \gets x3$
    \EndIf
    \State $x1 \gets x1 + 2$
\EndWhile
\end{algorithmic}

Анализируя исходный текст программы значение в регистре x31 в конце выполнения программы равно числу 8 (максимальное число)

Дизассемблированная код  программы выше (\ref{lst:task1}) находится в листинге \ref{lst:task1_disasm}:

\begin{lstlisting}[style=asm, caption=Дизассемблированный код программы для задания 1, label=lst:task1_disasm]
        SYMBOL TABLE:   0005                    c.addi  x0,1
        80000000 l    d  .text  00000000 .text  c.unimp
        80000038 l    d  .data  00000000 .data  c.slli  x0,0x1
        00000000 l    df *ABS*  00000000 task.o c.unimp
        00000009 l       *ABS*0700000000 len    .4byte  0x7
        00000002 l       *ABS*  00000000 enroll .2byte  0x8
        00000004 l       *ABS*  00000000 elem_sz
        80000038 l       .dataop00000000 _x --reverse-bytes=4 test.elf test.bin
        80000014 l       .textbi00000000 lp
        80000024 l       .textt.00000000 lt1
        8000002c l       .text-500000000 lt22/riscv-lab/src | on main ?1  ls
        80000034 l       .text  00000000 lp2
        80000000 g       .textt.00000000 _start
        8000005c g       .data-500000000 _end/riscv-lab/src | on main ?1  cat test.hex
        ok
        00200a13
        00000097
        Disassembly of section .text:
        0000a103
        80000000 <_start>:
        80000000:       00000097                auipc   x1,0x0
        80000004:       03808093                addi    x1,x1,56 # 80000038 <_x>
        80000008:       02408a13                addi    x20,x1,36
        8000000c:       0000af83                lw      x31,0(x1)
        80000010:       00408093                addi    x1,x1,4
        002f8fb3
        80000014 <lp>:
        80000014:       0000a103                lw      x2,0(x1)
        80000018:       0040a183                lw      x3,4(x1)
        8000001c:       01f16463                bltu    x2,x31,80000024 <lt1>
        80000020:       00200fb3                add     x31,x0,x2
        00000001
        80000024 <lt1>:
        80000024:       01f1e463                bltu    x3,x31,8000002c <lt2>
        80000028:       00300fb3                add     x31,x0,x3
        00000005
        8000002c <lt2>:
        8000002c:       00808093                addi    x1,x1,8
        80000030:       ff4092e3                bne     x1,x20,80000014 <lp>
        ~/gith/bmstu-comparch-5th-sem/lab_02/riscv-lab/src | on main ?1
        80000034 <lp2>:
        80000034:       0000006f                jal     x0,80000034 <lp2>

        Disassembly of section .data:

        80000038 <_x>:
        80000038:       0001                    c.addi  x0,0
        8000003a:       0000                    c.unimp
        8000003c:       0002                    c.slli64        x0
        8000003e:       0000                    c.unimp
        80000040:       00000003                lb      x0,0(x0) # 0 <enroll-0x2>
        80000044:       0004                    .2byte  0x4
        80000046:       0000                    c.unimp
        80000048:       0008                    .2byte  0x8
        8000004a:       0000                    c.unimp
        8000004c:       0006                    c.slli  x0,0x1
        8000004e:       0000                    c.unimp
        80000050:       00000007                .4byte  0x7
        80000054:       0005                    c.addi  x0,1
        80000056:       0000                    c.unimp
        80000058:       0004                    .2byte  0x4
        ...
\end{lstlisting}
\clearpage

\section{Задание 2}

В данной секции представлены результаты выполнения выборки и диспетчеризации.\newline

В соответствии с таблицей, приведенной в репозитории, необходимо было получить снимок экрана, содержащий
временную диаграмму выполнения стадий выборки и диспетчеризации команды с
указанным адресом (Вариант №18, адрес - 80000024, 2-я итерация). Рисунок \ref{img:wave} показывает результат выполнения.

\includeimage
{wave} % Имя файла без расширения (файл должен быть расположен в директории inc/img/)
{f} %
{h} %
{0.75\textwidth} % Ширина рисунка
{Диспетчирезация команды с адресом 80000024} % Подпись рисунка
\clearpage

\section{Задание 3}

В данной секции представлены результаты выполнения декодирования и планирования.\newline

В соответствии с таблицей, приведенной в репозитории, необходимо получить снимок экрана, содержащий
временную диаграмму выполнения стадии декодирования и планирования на выполнение
команды с указанным адресом (Вариант №18, адрес - 80000030, 2-я итерация).
Рисунок \ref{img:wave2} показывает результат выполнения.

\includeimage
{wave2} % Имя файла без расширения (файл должен быть расположен в директории inc/img/)
{f} %
{h} %
{0.75\textwidth} % Ширина рисунка
{Декодирование команды с адресом 80000030} % Подпись рисунка

Конфликта нет
\clearpage

\section{Задание 4}

В соответствии с таблицей, приведенной в репозитории, получить снимок экрана, содержащий
временную диаграмму выполнения стадии выполнения
команды с указанным адресом (Вариант №18, адрес - 8000001с, 2-я итерация).
Рисунок \ref{img:wave3} показывает результат выполнения.

\includeimage
{wave3} % Имя файла без расширения (файл должен быть расположен в директории inc/img/)
{f} %
{h} %
{1\textwidth} % Ширина рисунка
{Выполнение команды с адресом 8000001с} % Подпись рисунка
\clearpage

\section{Задание 5}

\subsection{Результаты выполнения}
В данной секции представлены результаты выполнения программы в соответствии с вариантом.

Рисунок \ref{img:task1} показывает результат значение регистра x31 после выполнения программы,
что соответствует ранее полученному результату.

\includeimage
{task1} % Имя файла без расширения (файл должен быть расположен в директории inc/img/)
{f} %
{h} %
{0.75\textwidth} % Ширина рисунка
{Значение регистра x31} % Подпись рисунка

Далее представлены временные диаграммы сигналов, соответствующих всем
стадиям выполнения команды в соответствии с вариантом.

А именно:

\begin{itemize}
\item Рисунок \ref{img:task2_fetch} показывает результат выполнения стадии выборки и диспетчирезации.
\includeimage
{task2_fetch} % Имя файла без расширения (файл должен быть расположен в директории inc/img/)
{f} %
{h} %
{0.75\textwidth} % Ширина рисунка
{Выборка и диспетчирезация} % Подпись рисунка

\item Рисунок \ref{img:task2_exec} показывает результат выполнения стадии декодирования и выполнения.
\includeimage
{task2_exec} % Имя файла без расширения (файл должен быть расположен в директории inc/img/)
{f} %
{h} %
{0.75\textwidth} % Ширина рисунка
{Декодирование и выполнение} % Подпись рисунка

\item Рисунок \ref{img:task2_exec_lsu} показывает результат выполнения стадии выполнения в блоке LSU.
\includeimage
{task2_exec_lsu} % Имя файла без расширения (файл должен быть расположен в директории inc/img/)
{f} %
{h} %
{0.75\textwidth} % Ширина рисунка
{Декодирование и выполнение} % Подпись рисунка
\end{itemize}

Таблица \ref{img:table1} показывает трассу выполнения программы.
\includeimage
{table1}
{f}
{h}
{1\textwidth}
{Трасса неоптимизированной программы}
\clearpage

\subsection{Оптимизация программы}

В данной секции представлены выводы об эффективности программы, а также возможных путях ее улучшения.

Анализируя трассу, мы можем увидеть, что двойноное условие внутри цикла создает множество конфликтов ветвления,
что приводит к тому, что ветвление не может быть корректно предсказано.
Сделаем предположение, что будет лучше, мы будем использовать одно условие внутри цикла (и загрузку одной переменной).

В результате (рисунок \ref{img:optim}) мы получим следующее время выполнения программы:

\includeimage
{unoptim}
{f}
{h}
{1\textwidth}
{Версия программы с плохой оптимизацией}
\clearpage

Время выполнения увеличилось с 70 тиков до 82, предположение оказалось неверным. Код неудачной программы прилагается в листинге \ref{lst:unoptim}.

\lstinputlisting[style=asm, caption=Программа для задания 6 (вар.18; неопт.), label={lst:unoptim}]{../riscv-lab/src/task_unoptimized.s}

Вглядываяь более внимательно, мы можем заметить, что внутри цикла происходит конфликт между операциями загрузки и ветвления.
Попробуем поместить другое вычисление между этими двумя операциями (в данном случае увеличение регистра, отвечающего за индексацию, который стоял в конце цикла).

В результате (рисунок \ref{img:optim}) мы получим следующее время выполнения программы:

\includeimage
{optim}
{f}
{h}
{1\textwidth}
{Версия программы с хорошей оптимизацией}
\clearpage
Время выполнения уменьшилось с 70 тиков до 66, предположение оказалось верным. Код удачной программы прилагается в листинге \ref{lst:optim}.

\lstinputlisting[style=asm, caption=Программа для задания 6 (вар.18; опт.), label={lst:optim}]{../riscv-lab/src/task_optimized.s}

Таблица \ref{img:table2} содержит трассу выполнения программы с оптимизацией.

\includeimage
{table2}
{f}
{h}
{1\textwidth}
{Трасса выполнения программы с оптимизацией}
\clearpage

\section{Выводы}

В ходе выполнения лабораторной работы были изучены основные архитектурные особенности процессора RISC-V, а также основные инструкции и регистры процессора.
Также удалось оптимизировать программу, умменьшив время выполнения с 70 тиков до 66 тиков за счет уменьшения количества конфликтов