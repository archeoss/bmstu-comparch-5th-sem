\chapter{Исследования расслоения динамической памяти}

Исходные данные: размер линейки кэш-памяти верхнего уровня (64Б), объем физической памяти(16Гб).

Настраиваемые параметры: максимальное расстояния между  читаемыми блоками (64К),
шаг увеличения расстояния между читаемыми 4-х байтовыми ячейками (64Б),
размер массивами (4М).

Графики полученных характеристик представлены на рисунке \ref{img:ex1_1}.

\includeimage
{ex1_1}
{f}
{h}
{1\textwidth}
{Результат исследования расслоения динамической памяти}

График показывает количество тактов работы алгоритма.
Ось абсцисс отражает шаг приращения адреса читаемых данных.
Ось ординат отображает количество тактов.

Следует учесть, что шаг чтения, меньший размера линейки кэш-памяти,
приводит к получению результирующей кривой,
имеющей пилообразный характер:
каждое второе обращение будет выполняться не к динамической памяти, а к кэш-памяти.
Это явление показано на рисунке \ref{img:ex1_2}.

\includeimage
{ex1_2}
{f}
{h}
{1\textwidth}
{Результат исследования расслоения динамической памяти с меньшим шагом увеличения расстояния между читаемыми 4-х байтовыми ячейками (32Б)}

Результаты эксперимента:количество банков динамической памяти, размер одной страницы динамической памяти, количество страниц в динамической памяти.

По графику можно определить несколько параметров.
\begin{itemize}
    \item Минимальный шаг чтения динамической памяти, при котором происходит постоянное обращение к одному и тому же банку, соответствует первому локальному экстремуму полученной функции (точка Т1=1024 байт).
По полученному значению шага Т1 можно определить количество банков памяти: Б = Т1/П = 1024/128=8, где П – объем данных, являющийся минимальной порцией обмена кэш-памяти верхнего уровня с оперативной памятью и соответствует размеру линейки кэш-памяти верхнего уровня.
    \item При достижения глобального экстремума, после которого рост локальных экстремумов не происходит, определяется характерная точка T2 (4096 байт).
    Соответствующи данной точке шаг чтения является наихудшим при обращении к динамической памяти, т.к. приводит к постоянному закрытию и открытию страниц динамической памяти.
    Таким образом шаг Т2 соответствует расстоянию (в байтах) между началом двух последовательных страниц одного банка.
    Зная количество банков, определяем размер одной страницы: РС = Т2/Б = 4096/8=512.
    \item Зная параметры РC и Б, а также полный объем памяти О определяем количество страниц физической оперативной памяти: С = О/(РС*Б*П)=O/(512*8*128)
\end{itemize}

\section*{Вывод}
Оперативная память расслоена и неоднородна, поэтому обращение к последовательно расположенным данным требует различного времени из-за наличия открытыя и закрытыя
страниц динамической памяти.
При этом чем больше адресное расстояние, тем больше время доступа.
В связи с этим для создания эффективных программ необходимо учитывать расслоение памяти и размещать рядом данные для непосредственной обработки.


\chapter{Сравнение эффективности ссылочных и векторных структур}

Настраиваемые параметры: количество элементов в списке (4М), максимальная фрагментации списка (256К), шаг увеличения фрагментации (4К).

Графики полученных характеристик представлены на рисунке \ref{img:ex2}.

\includeimage
{ex2}
{f}
{h}
{1\textwidth}
{Результат исследования эффективности ссылочных и векторных структур}

Красный график (верхний) показывает количество тактов работы алгоритма, использующего список.
Зеленый график (нижний) показывает количество тактов работы алгоритма, использующего массив.
Ось абсцисс отражает фрагментацию списка.


Результаты эксперимента: отношение времени работы алгоритма, использующего
зависимые данные, ко времени обработки аналогичного алгоритма обработки независимых
данных = 54,2


\section*{Вывод}
Видна проблема семантического разрыва: машина не присоблена к работе со ссылочными структурами.
Использовать структуры данных надо с учётом скрытых технологических констант.
Если алгоритм предполагает возможность использования массива, а списки не дают существенной разницы, то использование массива вполне оправдано.


\chapter{Исследование эффективности программной предвыборки}

Исходные данные: степень ассоциативности и размер TLB данных.

Настраиваемые параметры: шаг увеличения расстояния между читаемыми данными (1024Б), размер массиваи (256K).

Графики полученных характеристик представлены на рисунке \ref{img:ex3}.

\includeimage
{ex3}
{f}
{h}
{1\textwidth}
{Результат  исследования эффективности предвыборки}

Красный график (верхний с острыми пиками) показывает количество тактов работы алгоритма без предвыборки.
Зеленый график (нижний без значимых пиков) показывает количество тактов работы алгоритма с использованием предвыборки.
Ось абсцисс отражает смещение читаемых данных от начала блока.

Результаты эксперимента: отношение времени последовательной обработки блока
данных ко времени обработки блока с применением предвыборки =  2,3815041;
количество тактов первого обращения к странице данных = 16400

\section*{Вывод}
Обработка больших массивов информации сопряжена с открытием большого количества физических страниц памяти.
При первом обращении к странице памяти наблюдается увеличенное время доступа к данным в 20 раз,
так как оно при отсутствии информации в TLB вызывает двойное обращение к оперативной памяти:
сначала за информацией из таблицы страниц, а далее за востребованными данными.
Поэтому для ускорения работы программы можно использовать предвыборку.
Например, пока процессор занят некоторыми расчетами и не обращается к памяти,
можно заблаговременно провести все указанные действия благодаря дополнительному запросу небольшого количества данных из оперативной памяти.

Также стоит стараться не использовать в программе массивы, к которым обращение выполняется только один раз.

\chapter{Исследование способов эффективного чтения оперативной памяти}

Исходные данные: адресное расстояние между банками памяти, размер буфера чтения.

Настраиваемые параметры: размер массива (2М), количество потоков данных (64).

Графики полученных характеристик представлены на рисунке \ref{img:ex4}.

\includeimage
{ex4}
{f}
{h}
{1\textwidth}
{Результат  исследования оптимизирующих структур данных}

Красный график (верхний) показывает количество тактов работы алгоритма, использующего неоптимизированную структуру.
Зеленый график (нижний) показывает  количество тактов работы алгоритма с использованием оптимизированной структуры.
Ось абсцисс отражает количество одновременно обрабатываемых массивов.

Результаты эксперимента: отношение времени обработки блока памяти неоптимизированной структуры ко времени обработки блока структуры,
обеспечивающей эффективную загрузку и параллельную обработку данных = 1,1428789.

\section*{Вывод}
Эффективная обработка нескольких векторных структур данных без их дополнительной оптимизации не использует в должной степени возможности аппаратных ресурсов.

Для создания структур данных, оптимизирующих их обработку, необходимо передавать в каждом пакете только востребованную для вычислений информацию.
То есть для ускорения алгоритмов необходимо правильно упорядочивать данные.

\chapter{Исследование конфликтов в кэш-памяти}

Исходные данные: размер банка кэш-памяти данных первого (320K) и второго(5M) уровня, степень ассоциативности кэш-памяти первого и второго уровня, размер линейки кэш-памяти первого и второго уровня.

Настраиваемые параметры: размер банка кэш-памяти (256К), размер линейки кэш-памяти (128б), количество читаемых линеек (32).

Графики полученных характеристик представлены на рисунке \ref{img:ex5}.

\includeimage
{ex5}
{f}
{h}
{1\textwidth}
{Результат  исследования конфликтов в кэш-памяти}

Красный график (верхний) показывает  количество тактов работы процедуры, читающей данные с конфликтами в кэш-памяти.
Зеленый график (нижний) показывает количество тактов работы процедуры, не вызывающей конфликтов в кэш-памяти.
Ось абсцисс отражает смещение читаемой ячейки от начала блока данных

Результаты эксперимента: отношение времени обработки массива с конфликтами в кэш-памяти ко времени обработки массива без конфликтов = 3,499684 .

\section*{Вывод}
Попытка читать данные из оперативной памяти с шагом, кратным размеру банка, приводит к их помещению в один и тот же набор.
Если же количество запросов превосходит степень ассоциативности кэш-памяти, т.е. количество банков или количество линеек в наборе,
то наблюдается постоянное вытеснение данных из кэш-памяти, причем больший ее объем остается незадействованным.
Кэш память ускоряет работу процессора в 3.5 раз


\chapter{Сравнение алгоритмов сортировки}

Исходные данные: количество процессоров вычислительной системы(4), размер пакета, количество элементов в массиве, разрядность элементов массива

Настраиваемые параметры: количество 64-х разрядных элементов массивов (2М), шаг увеличения размера массива (64К).

Графики полученных характеристик представлены на рисунке \ref{img:ex6}.
\includeimage
{ex6}
{f}
{h}
{1\textwidth}
{Результат  исследования алгоритмов сортировки}

Фиолетовый график (верхний) показывает количество тактов работы алгоритма QuickSort.
Красный график (средний) показывает  количество тактов работы неоптимизированного алгоритма Radix-Counting.
Зеленый график (нижний) показывает количество тактов работы оптимизированного под 8-процессорную вычислительную систему алгоритма Radix-Counting.
Ось абсцисс отражает количество 64-разрядных элементов сортируемых массивов.

Результаты эксперимента: отношение времени сортировки массива алгоритмом QuickSort ко времени сортировки алгоритмом Radix-Counting Sort (=2,6370175)
и ко времени сортировки Radix-Counting Sort, оптимизированной под 8-процессорную вычислительную систему (=3,7621133).

\section*{Вывод}
Существует алгоритм сортировки менее чем линейной вычислительной сложности.

\chapter*{Общий вывод}
\addcontentsline{toc}{chapter}{Заключение}

В результате выполнения лабораторной работы были изучены принципы эффективного использования подсистемы памяти современных универсальных ЭВМ.

В ходе работы также
\begin{itemize}
    \item был проработан теоретический материал, касающийся особенностей функционирования подсистемы памяти современных конвейерных суперскалярных ЭВМ;
    \item были изучены возможности программы PCLAB;
    \item изучены средства идентификации микропроцессоров;
    \item проведены исследования времени выполнения тестовых программ;
    \item сделаны выводы об архитектурных особенностях используемых ЭВМ.
\end{itemize}
