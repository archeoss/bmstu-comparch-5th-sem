\section*{ВВЕДЕНИЕ}
\addcontentsline{toc}{section}{ВВЕДЕНИЕ}
Практикум посвящен освоению принципов работы вычислительного комплекса Тераграф и получению практических навыков решения задач обработки множеств на основе гетерогенной вычислительной структуры. В ходе практикума необходимо ознакомиться с типовой структурой двух взаимодействующих программ: хост-подсистемы и программного ядра sw\_kernel. Участникам предоставляется доступ к удаленному серверу с ускорительной картой и настроенными средствами сборки проектов, конфигурационный файл для двухъядерной версии микропроцессора Леонард Эйлер, а также библиотека leonhard x64 xrt c открытым исходным кодом.
